\documentclass[]{article}
\usepackage[left=2cm,right=4cm,top=1cm,bottom=2cm]{geometry}
\usepackage[utf8]{inputenc}
%\usepackage[latin1]{inputenc}
\usepackage[italian]{babel}
\usepackage{amsmath}
\usepackage{amsfonts}
\usepackage{amssymb}
\usepackage{makeidx}
\usepackage{graphicx}
\usepackage{lmodern}
\usepackage{enumitem}
%\usepackage{color}}
%opening
\title{Il software libero - questioni di etica }

\author{Salvatore D'Asta}

\begin{document}

\maketitle

\begin{abstract}
In ambito informatico il termine software ha assunto un'importanza fondamentale. Il software, cioè i programmi, che non sono altro che  la sequenza di istruzioni che  un computer deve eseguire per poter volgere un determinato compito, assumono un ruolo fondamentale in informatica. La macchina \textit{"Computer"}, intendendo con questo termine l'insieme fisico di circuiti e microprocessori ecc., non avrebbe motivo di esistere senza il software. La società moderna è ormai, e sempre più lo sarà in futuro, dominata dai computer, la cosiddetta \textit{intelligenza artificiale} viene sempre più spesso invocata per risolvere problemi che coinvolgono la nostra economia, ed altri aspetti importanti della nostra società, i nostri dati sono il \textit{"petrolio del 21-esimo secolo"}. Ma chi controlla l'\textit{intelligenza artificiale} e i nostri dati? Chi sono i proprietari degli algoritmi che determinano le decisioni sulle nostre vite? Attualmente i colossi dell'informatica detengono la proprietà della maggior parte dei nostri dati e dei software che utilizziamo: il cosiddetto \textit{software proprietario}. Ma è sicuro lasciare in mano di multinazionali private decisioni che riguardano il nostro futuro? Ma d'altra parte esiste un'alternativa? Sì, ad esempio il \textit{software Libero}.       
\end{abstract}


\section{Software\textit{ "aperto"} e software \textit{"chiuso"}}
Per semplificare il più possibile possiamo classificare i software in due grandi categorie: \textbf{software aperto} e \textbf{software chiuso}. Al primo appartengono, ad esempio, i Sistemi Operativi \textbf{GNU/Linux}, la suite per l’ufficio \textbf{Libre Office} ecc.,  al software chiuso appartengono, ad esempio, tutti i Sistemi operativi proprietari ( \textbf{Apple}, \textbf{Microsoft}, ecc.) e gli applicativi prodotti dagli stessi (\textbf{Word}, \textbf{Power point} …). \\  

La differenza sostanziale tra le due categorie sta nel fatto che \textbf{del software chiuso non è dato sapere nulla} riguardo a cosa esso contenga diversamente del \textbf{software aperto} che fornisce il codice sorgente e \textbf{del quale si può sapere tutto}. \\

Per fare un analogia supponendo che il software sia un piatto ordinato al ristorante il software  aperto verrà servito con allegata la ricetta completa di ingredienti e procedimento. Il software chiuso no.

\section{Il software libero}
Il software libero è un software  aperto che rispetta le 4 leggi fondamentali proposte dalla\\
 \textbf{F.S.F. (Free Software Foundation)}:
\begin{itemize}
	\item[$ \rightarrow $] Legge 0:
	\item[$ \rightarrow $] Leggi 1:
	\item[$ \rightarrow $] Legge 2:
	\item[$ \rightarrow $] Legge 3:
\end{itemize}

\section{Il software proprietario}
\section{La scuola dovrebbe adottare delle soluzioni etiche}
La scuola ha il compito di educare i ragazzi alla \textbf{condivisione} ed alla \textbf{cooperazione}, spingerli ad \textbf{essere curiosi}, \textbf{solidali} ed \textbf{aperti} nei confronti degli altri, a favorire condizioni di \textbf{pari opportunità} tra i gli alunni  ed  è proprio per questo che la scuola dovrebbe spingerli all’uso del software aperto.
\begin{enumerate}
	\item \textbf{Condivisione}:
	\begin{itemize}
		\item Il software aperto puoi condividerlo con gli altri (anzi sei vivamente invitato a farlo) 
		\item il software chiuso no e se lo fai commetti un reato.
	\end{itemize}
	\item \textbf{Essere curiosi}
	\begin{itemize}
		\item Il software aperto favorisce la naturale curiosità dei ragazzi mettendo a loro disposizione il codice sorgente (cioè la ricetta di come il software e stato creato). 
		\item Il software chiuso no. Non fornisce il codice sorgente
	\end{itemize}
	\item \textbf{Pari opportunità}
	\begin{itemize}
		\item Il software aperto, è spesso gratuito e può girare in modo soddisfacente anche su computer vecchi o economici. In questo modo anche le famiglie più povere possono permettersi la spessa per l'acquisto di un PC. 
		\item Il software chiuso è spesso a pagamento e necessita di macchine sempre più potenti e costose. Questo limita l'accesso alle tecnologie alle persone più ricche aumentando le disuguaglianze sociali. 
	\end{itemize}
	\item \textbf{Sostenibilità}
	\begin{itemize}
		\item L'uso del software aperto allunga la vita dei PC evitando che finiscano prematuramente in discarica.
		\item Il software chiuso costringe a \textit{buttar via} un PC perfettamente funzionante dopo pochi anni.  
	\end{itemize}
\end{enumerate}


\section{I termini dell'accordo}
Si tratta di un accordo in base al quale il Miur da in gestione a Microsoft le caselle di posta elettronica che passano ad “office 365” il quale si impegna ad organizzare corsi di formazione gratuiti (ovviamente su prodotti Microsoft).
\\In base all’art. 68. “Analisi comparativa delle soluzioni”\footnote{text} del  Codice dell’amministrazione digitale  gli istituti scolastici nella scelta dei prodotti software da usare a scuola non conteggeranno i costi della formazione del personale e questo li indurrà a scegliere prodotti Microsoft. 
\\\\
Ora se si trattasse di una fornitura di banchi e di sedie non sarebbe nulla di preoccupante, la cosa che invece mi allarma è che questa scelta coinvolge pesantemente la formazione dei nostri ragazzi ed ha dei \textbf{risvolti etici molto importanti}.

\end{document}
